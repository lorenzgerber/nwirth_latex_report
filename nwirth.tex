%%%%%%%%%%%%%%%%%%%%%%%%%%%%%%%%%%%%%%%%%%%%%%%%%%%%%%%%%%%%%%%%%%%%%%%%%%%%%%
% Detta är ett exempel på ett latexdokument.
% 
% Alla dokument består av följande delar:
%
%          \documentclass[optioner]{dokumentklass}
%            ...inställningar...
%          \begin{document}
%            ...text...
%          \end{document}
%
% Som ni kanske redan har förstått är används procent (%) för
% kommentarer.
%%%%%%%%%%%%%%%%%%%%%%%%%%%%%%%%%%%%%%%%%%%%%%%%%%%%%%%%%%%%%%%%%%%%%%%%%%%%%%

\documentclass[a4paper]{article}
\usepackage[utf8]{inputenc}
\usepackage[T1]{fontenc}                % För svenska bokstäver
\usepackage[swedish]{babel}             % För svensk avstavning och svenska
                                        % rubriker (t ex ``innehållsförteckning)
\title{Datavetaren Niklaus Emil Wirth}
\author{Lorenz Gerber}
\date{}           % Blir dagens datum om det utelämnas

\begin{document}

\maketitle                      % Skriver ut rubriken som vi
                                % deklarerade ovan med \title, \author
                                % och eventuellt \date

\section{Introduktion}          % Detta kommando gör en rubrik

Niklaus E. Wirth är en få icke amerikanska datavetare som har vunnit
`Turingpriset', en utmärkelse som är nämnd efter den engelska datavetar 
pionären Allan Turing. Niklaus With är född och uppvuxen i Schweiz. Han fick 
`Turingpriset' 1984 för att ha utvecklad flera nya, innovativa programmeringsspråk.

\section{Utbilding och Karriär}

Winterthur, staden där Niklaus Wirth är född och uppvüxen ligger inte långt ifrån
staden Zurich, där han sen studerade elektronik ingejör på den anrika
tekniska högskolan `ETH'. Efter ingenjörsexamen läste Niklaus Wirth till 
en MSc på Laval Universitet i Kanada. Han doktorerade i Berkley (University of 
California). Han var assistent professor först på Stanford Universitet och
senare tilbaks i Schweiz på Zurich Universitet. 1968 fick han en professor
tjänst i samma institut som han började sin akademisk utbildning, den `Eidgenössische
Technische Hochschule' Zurich.  

\section{Turingpriset}
Programmeringsspåket Pascal är kanske skapelse som Niklaus Wirth är mest känd
för. Språket var 

\end{document}                 % The input file ends with this command.
